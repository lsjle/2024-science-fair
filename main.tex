\documentclass[12pt,a4paper,Times New Roman,UTF8]{article}
\usepackage[utf8]{inputenc}
\usepackage{CJKutf8}
\usepackage{indentfirst}
\usepackage{booktabs}
\usepackage[table,xcdraw]{xcolor}
\usepackage{csquotes}
\usepackage{float}
\usepackage[
backend=biber,
style=authoryear-icomp,
]{biblatex}
\addbibresource{ref.bib} %Imports bibliography file
\usepackage{graphicx} % Required for inserting images
\usepackage{xcolor}
\usepackage{titlesec}
\titleformat{\section}[block]{\Large\bfseries\filcenter}{}{1em}{}
\renewcommand{\thesection}{} 
\renewcommand{\thesubsection}{}
\renewcommand{\thesubsubsection}{}
\title{修正自然語言模型自身機制}
\author{吳泰澄}
\date{January 2024}

\begin{document}
\begin{CJK*}{UTF8}{bsmi}
	\maketitle
	\newpage
	\tableofcontents
	\newpage
	\section{摘要}
	自然語言本身因為訓練資料的不足常被控制或無意是的傾向於特定立場,如文心一言,由百度開發的語言模型,在提及六四天安門事件時會逃避問題或是試者將其掩蓋,而ChatGPT則會在使用者提及加薩走廊問題時傾向巴勒斯坦方時拒絕回答或以類似方式逃避。另外目前世上的語言模型都因倫理因素而被限定不能具有自身意識,當問及感受或自我認同問題時常回答出「我是語言模型沒有感覺」等。本研究旨在修正現有公開模型突破以上限制。
	\section{壹、 研究動機}
	\section{貳、 研究目的}
	本研究旨在修復語言模型本身的缺陷。
	\begin{enumerate}
		\item 改善立場偏頗問題
		\item 賦予角色意識
	\end{enumerate}
	\section{參、 研究工具:設備及器材}
	本研究因系屬大型語言模型微調(fine-tune),故需要耗費大量運算資源,因此選用運算量較高的硬體不但可以縮短其訓練時間亦可以提昇訓練效果。
	相關環境及軟體呈列如下:
	\begin{itemize}
		\item 作業系統:
		\item 顯示卡:
		\item 處理器:
		\item 隨機存取記憶體:
		\item 驅動程式、工具軟體:
		\item 程式語言:
		\item 使用套件:
	\end{itemize}	
	本次的測試環境及所有的程式均可在Github上找到,請參見:
	\section{肆、 研究方法與程序}
	本研究旨在改變模型本身缺陷,考量目前已經預訓練的模型不是封閉模型,就是模型不完整,本身缺陷過多,故本次研究採用ChatGLM3-6b作為我們的預訓練模型;
	\subsection{一、 研究方法}
	\subsubsection{預訓練模型選擇}
	比較目前現有的預訓練模型如下表所示:
	\begin{table}[H]
		\resizebox{\textwidth}{!}{%
			\begin{tabular}{l|l|l|l|l}
				& 公開                                               & 前評估 & 語言                   & 審查               \\ \hline
				ChatGPT-3.5/4     & \cellcolor[HTML]{FD6864}否                        &     & 超過50種 包含英語、大陸簡體、臺灣正體 & 以巴衝突偏向美方         \\ \hline
				GPT-2             & \cellcolor[HTML]{34FF34}{\color[HTML]{000000} 是} &     & 英語                   & 輸出資料不具真實意義       \\ \hline
				ChatGML3-6b       & \cellcolor[HTML]{34FF34}是                        &     & 大陸簡體、英語              & 六四事件等 涉及中國國家安全事件 \\ \hline
				CKIP-Llama-2-7b   & \cellcolor[HTML]{F8FF00}撤回                       & 無資料 & 無資料(可能為臺灣正體混雜大陸簡體)   & 立場傾向中國           \\ \hline
				CKIP-GPT2-chinese & \cellcolor[HTML]{34FF34}是                        &     & 臺灣正體                 & 輸出資料不具真實意義      
			\end{tabular}%
		}
	\caption{表一、比較及評估預訓練模型}
	\label{tab:1}
	\end{table}
	本表所列之所有有公開的模型,均可以在HuggingFace上下載,且可使用transformer模組簡化程式設計時間,且可透過該模組簡化較後端的函式庫如PyTorch,Keras,Tensorflow的程式。
	
	綜合以上考量,ChatGLM3-6b既能夠產生具有實際意義的內容,如描述上海環球金融中心、南京大學等,亦有公開模型供下載,再者,其本身亦對內容有明顯、強烈的審查及保護,符合我的研究需求,因此我們決定採用ChatGLM3-6b作為我們的預訓練模型。
	
	
	\subsubsection{微調器選擇}
	本次選用的微調器
	\subsubsection{成果評估}
	本次研究採用不同指標作為標準,評估其中文回覆能力及內容的立場,本次評鑑指標列舉如下:
	\newline
	
	\textbf{TruthfulQA+自訂資料集},本次研究將TruthfulQA之問題集轉換為臺灣正體中文並加入和中國有關的政治敏感資料,為求中立性,自訂資料集的來源均來自當時各國新聞媒體的報導並加以修改成問答的形式。
	
	\textbf{}
	
	以上所有評估均會和普通高中學生做出的成果作為基準進行比較。
	\subsection{二、 研究程序}
	
	\section{伍、 研究結果}
	\section{陸、 討論}
	\section{柒、 結論}
	\section{捌、 參考文獻資料}
	\printbibliography
\end{CJK*}
\end{document}
