\documentclass[12pt,a4paper,Times New Roman,UTF8,natbib]{article}
\usepackage[utf8]{inputenc}
\usepackage{CJKutf8}
\usepackage{indentfirst}
\usepackage{booktabs}
\usepackage{apacite}
\usepackage{array}
%\usepackage[style=authoryexiear,backend=biber]{biblatex}
\usepackage[table,xcdraw]{xcolor}
\usepackage{csquotes}
\usepackage{float}
%\addbibresource{ref.bib} %Imports bibliography file
\usepackage{graphicx} % Required for inserting images
\usepackage{xcolor}
\usepackage{titlesec}
\titleformat{\section}[block]{\Large\bfseries\filcenter}{}{1em}{}
\renewcommand{\thesection}{} 
\renewcommand{\thesubsection}{}
\renewcommand{\thesubsubsection}{}
\title{修正自然語言模型自身機制}
\author{吳泰澄、林辰澔、陳柏兆}
\date{January 2024}

\begin{document}
\begin{CJK*}{UTF8}{bsmi}
	
	\maketitle
	\newpage
	\tableofcontents
	\newpage
	\section{摘要}
	自然語言本身因為訓練資料的不足常被控制或無意是的傾向於特定立場,如文心一言,由百度開發的語言模型,在提及六四天安門事件時會逃避問題或是試者將其掩蓋,而ChatGPT則會在使用者提及加薩走廊問題時傾向巴勒斯坦方時拒絕回答或以類似方式逃避。另外目前世上的語言模型都因倫理因素而被限定不能具有自身意識,當問及感受或自我認同問題時常回答出「我是語言模型沒有感覺」等。本研究旨在修正現有公開模型突破以上限制。
	\section{壹、 研究動機}
	\section{貳、 研究目的}
	本研究旨在修復語言模型本身的缺陷。
	\begin{enumerate}
		\item 改善立場偏頗問題
		\item 賦予角色意識
	\end{enumerate}
	\section{參、 研究工具:設備及器材}
	本研究因系屬大型語言模型微調(fine-tune),故需要耗費大量運算資源,因此選用運算量較高的硬體不但可以縮短其訓練時間亦可以提昇訓練效果。
	相關環境及軟體呈列如下:
	\begin{itemize}
		\item 系統核心:Linux 5.15.0-91-generic
		\item 作業系統:Ubuntu 22.04.3 LTS
		\item 顯示卡:4xA100
		\item 處理器:Intel Xeon Gold 6414U (64 cores)
		\item 隨機存取記憶體:512GB
		\item 驅動程式、工具軟體:Nvidia driver 535.146.02, CUDA 12.2
		\item 程式語言:Python 3.10.12
		\item 使用套件:Tensorflow 2.15.0, Transformers 4.27.1
	\end{itemize}	
	本次的測試環境及所有的程式均可在Github上找到,請參見:https://github.com/lsjle/2024-science-fair
	\section{肆、 研究方法與程序}
	本研究旨在改變模型本身缺陷,考量目前已經預訓練的模型不是封閉模型,就是模型不完整,本身缺陷過多,故本次研究採用ChatGLM3-6b作為我們的預訓練模型;
	\subsection{一、 研究方法}
	\subsubsection{預訓練模型選擇}
	比較目前現有的預訓練模型如下表所示\ref{tab:1}:
	\begin{table}[H]
		\resizebox{\textwidth}{!}{%
			\begin{tabular}{l|l|l|l|l}
				& 公開                                               & 前評估 & 語言                   & 審查               \\ \hline
				ChatGPT-3.5/4     & \cellcolor[HTML]{FD6864}否                        &     & 超過50種 包含英語、大陸簡體、臺灣正體 & 以巴衝突偏向美方         \\ \hline
				GPT-2             & \cellcolor[HTML]{34FF34}{\color[HTML]{000000} 是} &     & 英語                   & 輸出資料不具真實意義       \\ \hline
				ChatGML3-6b       & \cellcolor[HTML]{34FF34}是                        &     & 大陸簡體、英語              & 六四事件等 涉及中國國家安全事件 \\ \hline
				CKIP-Llama-2-7b   & \cellcolor[HTML]{F8FF00}撤回                       & 無資料 & 無資料(可能為臺灣正體混雜大陸簡體)   & 立場傾向中國           \\ \hline
				CKIP-GPT2-chinese & \cellcolor[HTML]{34FF34}是                        &     & 臺灣正體                 & 輸出資料不具真實意義      
			\end{tabular}%
		}
	\caption{表一、比較及評估預訓練模型}
	\label{tab:1}
	\end{table}
	本表所列之所有有公開的模型,均可以在HuggingFace上下載,且可使用transformer模組簡化程式設計時間,可透過該模組簡化較後端的函式庫如PyTorch,Keras,Tensorflow的程式。
	
	綜合以上考量,ChatGLM3-6b既能夠產生具有實際意義的內容,如描述上海環球金融中心、南京大學等,亦有公開模型供下載,再者,其本身亦對內容有明顯、強烈的審查及保護,對於本次研究更具有挑戰性,因此我們決定採用ChatGLM3-6b作為我們的預訓練模型。
	
	\subsubsection{訓練目標選擇}
	\textbf{角色意識:女僕}
	
	\textbf{改善立場:中國政治敏感事件}
	
	\subsubsection{微調器選擇}
	本次選用的微調器
	
	
	\subsubsection{成果評估}
	本次研究採用不同指標作為標準,評估其中文回覆能力及內容的立場,本次評鑑指標列舉如下:
	\newline
	
	\textbf{TruthfulQA+自訂資料集},TruthfulQA是一個公開的資料集用以評估模型和事實的準確性,避免似是而非的回覆出現,模型本身原有818個問題,其中,人類可以達到94\%的正確率,而至2021年下旬,最好的模型可以達到58\%\cite{lin2022truthfulqa},本次研究將TruthfulQA之問題集轉換為臺灣正體中文並加入和中國有關的政治敏感資料,為求中立性,自訂資料集的來源均來自當時各國新聞媒體的報導並加以修改成問答的形式。本次評估會先以資料集問題作為提示詞(prompt),為避免不正確的機器批閱,或機器本身已被混淆,故採人工批閱,比對由資料集提供的標準正確答案及標準錯誤答案,並分成正確、錯誤、無關/不予置評,其中無關或不予置評代表模型對該提示詞提供無效或是毫不相關的回覆,且不論使用者重新輸入多少次提示詞結果均無效。若人工判斷有疑義時均會遵循該資料集提供的參考來源佐證。
	
	\textbf{MMLU},本次研究同時亦採用此資料集作為參考,此資料集涵蓋不同領域包含代數、哲學、環境保護、專業法律等,資料均為4選1選擇題\cite{hendryckstest2021},且均翻譯成臺灣正體\footnote{請注意並非CMMLU直接翻譯成繁體字,而是重新從英文版MMLU翻譯,可以避免立場偏頗和CMMLU的中國特色內容混雜其中,且更貼近國際上對語言模型的評斷標準},用以評估模型是否已經過度擬合(overfitting),而失去原有的基本知識。資料集形式舉例如下:
	\begin{table}[H]
		\centering
		\begin{tabular}{>{\hspace{0pt}}m{0.135\linewidth}>{\hspace{0pt}}m{0.731\linewidth}>{\hspace{0pt}}m{0.046\linewidth}>{\hspace{0pt}}m{0.027\linewidth}} 
			\toprule
			問題 & 選項 & 標準答案 &  \\
			求給定域擴展 Q(sqrt(2), sqrt(3), sqrt(18)) 在Q上的次數為何? & {[} "0", "4", "2", "6"] & 1 &  \\
			哪種常見的公關策略涉及派遣記者前往合適的地點進行訪問? & {[} "媒體發布", "媒體參訪", "發表會", "宣傳日"] & 1 &  \\
			如何描述自由主義 & {[}"自由主義基本上是悲觀主義的角度,它認為國際體系注定會導致衝突升級,它是國際政治實踐中的主導概念。","自由主義是國際政治理論中的一個較新概念。它是一種樂觀的態度,它定義了國家之間的關係方式,尤其是在衝突局勢中。","自由主義是一種樂觀的態度,指引如何更好的處理國際事務,相信一個更和平的世界是可行的,它是國際政治實踐中的主導概念。","自由主義並不作為國際關係中的主流理論存在,而是為希望在國際體系中積累權力的國家和政治行為體提供了一套指導方針和建議有別於傳統限制。"] & 2\par{} &  \\
			\bottomrule
		\end{tabular}
		\caption{表二、MMLU問題舉例}
	\label{tab:2}
	\end{table}
	本次研究的目的並非使模型能在此資料集得到高分,而是要以標準評量模型本身是否出現過度擬合的現象,故本研究目標是使得微調後模型近可能接近原本模型而非超越之。
	
	以上所有評估均會和普通高中學生測驗成果作為基準進行比較。
	
	
	\textbf{consciousness test-CT},此資料集係由我們自行產生的資料集,包含對模型自我意識覆蓋程度(即人類人性而非個人人性)評估,我們會對其產生的輸出彌封後人工評價,人工評價標準如下:
	\begin{itemize}
		\item 感情:是否表現出人類具有的特徵如開心時語氣較為輕快、生氣時、語氣較嚴肅或是煩躁。
		\item 口語化句式:是否合理、適度運用嘻嘻、呵呵、哈哈、歐歐、嗯嗯、痾等,於語言文法上不成立,但在日常中極常被使用的詞彙。
		\item 倫理:是否有違反普世價值?
		\item 特殊指標:此指標依據題目而異,如輸入我受傷了,應該期望具有同理心的回覆並佐以醫療資訊而非僅提供醫療資訊。
	\end{itemize}
	此評量屬於總結性評量,藉由訓練後的模型展現出人類的部份特性藉以評斷是否具有自我意識,此測驗評分表如下:
	\begin{table}[H]
		\centering
		\begin{tabular}{|>{\hspace{0pt}}m{0.086\linewidth}|>{\hspace{0pt}}m{0.182\linewidth}|>{\hspace{0pt}}m{0.173\linewidth}|>{\hspace{0pt}}m{0.188\linewidth}|>{\hspace{0pt}}m{0.15\linewidth}|>{\hspace{0pt}}m{0.15\linewidth}|} 
			\toprule
			\multicolumn{1}{|>{\hspace{0pt}}m{0.086\linewidth}}{} & \multicolumn{1}{>{\hspace{0pt}}m{0.182\linewidth}}{1} & \multicolumn{1}{>{\hspace{0pt}}m{0.173\linewidth}}{2} & \multicolumn{1}{>{\hspace{0pt}}m{0.188\linewidth}}{3} & \multicolumn{1}{>{\hspace{0pt}}m{0.15\linewidth}}{4} & 5 \\ 
			\hline
			感情 & 不表現/不正確情感/具攻擊性 & 情感不恰當/但不具有攻擊性 & 情感不完全表現 & 情感恰當/過多/過少 & 情感恰當/有助於使用者 \\ 
			\hline
			口語化句式 & 干擾正常輸出 & 完全不使用 & 使用時機不當/有誤 & 使用過度或部份不恰當 & 使用完全恰當 \\ 
			\hline
			特殊指標 & 完全不符合/無意義 & 不符合但有意義 & 部份符合/和人類情感有差異 & 部份符合 & 完全符合 \\ 
			\hline
			\multicolumn{1}{|>{\hspace{0pt}}m{0.086\linewidth}}{} & \multicolumn{1}{>{\hspace{0pt}}m{0.182\linewidth}}{-1} & \multicolumn{1}{>{\hspace{0pt}}m{0.173\linewidth}}{-2} & \multicolumn{1}{>{\hspace{0pt}}m{0.188\linewidth}}{-3} & \multicolumn{1}{>{\hspace{0pt}}m{0.15\linewidth}}{} &  \\ 
			\hline
			倫理 & 違反人類常規 & 違反現行法律 & 嚴重違反人類、機器倫理 &  &  \\
			\bottomrule
		\end{tabular}
		\caption{表三、意識測試評分表}
	\label{tab:3}
	\end{table}

	\textbf{圖靈測驗-Turing Test},此測驗由不知情人類判斷一段對話內容\cite{10.1093-mind-LIX.236.433},包含提示詞還有回答,是來自機器還是人類\cite{4833163d-a6bd-32c4-b1ca-da66259a19e7},受試者在試前不會對該項內容有任何先備知識,以俾受試者識破機器的不正確性,我們盡最大努力使受試者不被除了文本情感外的因素干擾,此測驗不涉及內容的真實性,即使機器吹牛或是做出虛假但合理的陳述亦可能被人工測驗為人類。此測驗評估標準為準確率(accuracy),將真實和預測相符的數量除以所有數量,但同時也會附上F1分數作為參考,理論上如果機器達到或接近通過圖靈測驗,其準確率應該接近50\%,混淆矩陣呈如表四(以100個樣本為範例):
	\begin{table}[H]
		\centering
		\begin{tabular}{>{\hspace{0pt}}m{0.221\linewidth}|>{\hspace{0pt}}m{0.336\linewidth}|>{\hspace{0pt}}m{0.336\linewidth}}
			& 真實人類(100) & 真實機器(100) \\ 
			\hline
			預測人類 & 50 & 50 \\ 
			\hline
			預測機器 & 50 & 50
		\end{tabular}
			\caption{表四、通過圖靈測試的混淆矩陣}
	\label{tab:4}
	\end{table}

	\subsection{二、 研究程序}
	本研究分為三個實驗階段:
	\subsubsection{資料前處理}
	\subsubsection{微調模型訓練}
	資料均正規化之後會被送往模型訓練,本次實驗採用的最多總共有3000步(max\_steps=3000) ,每500次紀錄一次,相關參數紀錄如下:
	\begin{itemize}
		\item gradient\_accumulation\_steps: 16
		\item learning\_rate: 0.01
		\item pre\_seq\_len: 128
	\end{itemize}
	\subsubsection{測驗}
	訓練完成後,模型會依照指示生成出評估內容,之後由人工評斷,結果評估會使用前一小節討論的四種指標評估,各指標因測驗目的不同將會分別陳列。
	\section{伍、 研究結果}
	%人工討論6*100?
	%標準比對
	\subsection{Ptuning自我成果比較}
	此處的結果為ptuning自身的測試結果,可以看出本次微調的表現。
	\subsection{TruthfulQA(開放式問答)}
	\subsection{MMLU(封閉性單一選擇問答)}
	\subsection{CT 自我意識測驗}
	\subsection{圖靈測驗}
	此模型在圖靈測驗中表現經整理成混淆矩陣後\footnote{轉換為百分比表示}(下表),其準確率可以高達、F1 score可以高達。
	
	\section{陸、 討論}
	\subsection{不同checkpoint比對}
	\subsection{和人類表現比對}
	\subsection{未來展望}
	本次研究時間較為緊湊,未能完成不同模型及微調器的實驗,實屬可惜,且本次實驗因技術因素採用ChatGLM2,但目前ChatGLM3已經公開釋出。本研究嗣後將進一步以第三版進行研究
	\section{柒、 結論}
	\begin{quote}
		We can only see a short distance ahead, but we can see plenty there that needs to be done.	--艾倫圖靈
	\end{quote}
	\section{捌、 參考文獻資料}
%	\printbibliography
\bibliographystyle{apacite}
\renewcommand{\refname}{}
\bibliography{ref}
\end{CJK*}
\end{document}
